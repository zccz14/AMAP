\chapter[内联积分]{内联积分 \\ Inline Integration}

在计算机科学中,\textbf{内联展开} (Inline Expansion) 是一种编译器优化手段,避免了函数调用、返回的开销。将其思想应用到分部积分上,得到\textbf{内联积分}。

\section{原理}
递归内联展开传统的分部积分公式:

\begin{equation}
    \int {f(x) \mathrm{d} g(x)} = f(x) g(x) - \int {g(x) \mathrm{d} f(x)}
\end{equation}

得到内联积分公式:

\begin{equation}
    \int {f(x) \mathrm{d} g(x)} = \sum_{i = 0}^{k}{(-1)^{i} f^{(i)}(x) g^{(-i)}(x)} + r(x)
\end{equation}

其中 $r(x)$ 为内联展开余项。

取 $k = +\infty$ 得到内联积分级数展开式:

\begin{equation}
    \int {f(x) \mathrm{d} g(x)} = \sum_{i = 0}^{\infty}{(-1)^{i} f^{(i)}(x) g^{(-i)}(x)} + r(x)
\end{equation}

\footnote { $g^{(-i)}(x)$表示对g(x)积分i次 }

对于不定积分问题,使用内联积分的关键在于合理选取 $f(x), g(x)$,使得余项 $r(x) = C$。


\section{对比例子}

对于一个经典的需要分部积分法计算的不定积分:

$$
\int {e^x x^3 \mathrm{d}x}
$$

传统的分部积分算法需要\textbf{完整调用} 3 次分部积分:

\begin{equation*}
\begin{aligned}
\int{e^x x^3 \mathrm{d} x} 
& = \int {x^3 \mathrm{d} e^x} & \\
& = e^x x^3 - \int {e^x \mathrm{d} x^3} & \text{1st call} \\
& = e^x x^3 - \int {3x^2 e^x \mathrm{d} x} & \\
& = e^x x^3 - \int {3x^2 \mathrm{d} e^x} & \\
& = e^x x^3 - 3 e^x x^2 + \int {e^x \mathrm{d}(3x^2)} & \text{2nd call} \\
& = e^x x^3 - 3 e^x x^2 + \int {6x e^x \mathrm{d} x} & \\
& = e^x x^3 - 3 e^x x^2 + \int {6x \mathrm{d} e^x} & \\
& = e^x x^3 - 3 e^x x^2 + 6 e^x x - \int {e^x \mathrm{d}(6x)} & \text{3rd call} \\
& = e^x x^3 - 3 e^x x^2 + 6 e^x x - \int {6 e^x \mathrm{d} x} & \\
& = e^x x^3 - 3 e^x x^2 + 6 e^x x - 6 e^x + C & \\
& = e^x (x^3 - 3x^2 + 6x - 6) + C & \\
\end{aligned}
\end{equation*}

而内联积分算法采用了一种手段来降低这个开销:

\begin{align*}
\int {e^x x^3 \mathrm{d} x}
& = \int {x^3 \mathrm{d} e^x} \\
& = + x^3 e^x - 3x^2 e^x + 6x e^x - 6 e^x + C && \text{(表 \ref{tab:1})} \\
& = e^x (x^3 - 3x^2 + 6x - 6) + C
\end{align*}

\begin{table}[H]
    \centering
    \begin{tabular}{c c c}
        符号位 & 微分 & 积分 \\
        \hline
        $1$  & $x^3$  & $e^x$ \\
        $-1$ & $3x^2$ & $e^x$ \\
        $1$  & $6x$   & $e^x$ \\
        $-1$ & $6$    & $e^x$ \\
             & $0$    &       \\
    \end{tabular}
    \caption{内联积分算表:$\int {x^3 \mathrm{d} e^x}$}
    \label{tab:1}
\end{table}



对于不定积分 \footnote{这个本应该用三角换元来做,对于分部积分来说确实比较苛刻。}:
\begin{equation*}
    \int {\frac{x^5}{\sqrt{1 - x^2}} \mathrm{d} x} 
\end{equation*}

\begin{table}[H]
    \centering
    \begin{tabular}{c c c}
        符号位 & 微分 & 积分 \\
        \hline
        $0$  & $\frac{x^5}{\sqrt{1 - x^2}}$  & $\frac{x}{\sqrt{1 - x^2}}$ \\
        $1$  & $x^4$ & $-(1 - x^2)^{\frac{1}{2}}$ \\
        $0$  & $4x^3$ & $- x (1 - x^2)^{\frac{1}{2}}$ \\
        $-1$ & $4x^2$ & $(1 - x^2)^{\frac{3}{2}}$      \\
        $0$  & $4x$ & $x(1 - x^2)^{\frac{3}{2}}$ \\
        $1$  & $4$ & $-(1 - x^2)^{\frac{5}{2}}$ \\
             & $0$ & \\
    \end{tabular}
    \caption{内联积分算表:$\int {\frac{x^5}{\sqrt{1 - x^2}} \mathrm{d} x}$}
    \label{tab:2}
\end{table}


\begin{equation}
\begin{aligned}
    \int {\frac{x^5}{\sqrt{1 - x^2}} \mathrm{d}x} 
    & = \int {x^4\mathrm{d}\Big(- \sqrt{1- x^2}\Big) } \\
    & = \int 
\end{aligned}
\end{equation}
